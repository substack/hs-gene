\documentclass[11pt,a4paper]{article}
\usepackage{listings}
\author{James Halliday}
\title{
    Research Proposal \\
    \normalsize{Typed Genetic Programming Ecosystem}
}

\begin{document}
\maketitle

Genetic programming is an approach to search and optimization where
a population of computer programs competes to maximize a fitness landscape.
As in biological evolution, these individuals undergo selection, variation, and
inheritance in order to produce an increasingly fit population over time.
Genetic programming achieves mutations through operations on abstract syntax
trees, which represent source code structure at a higher level.

Researchers such as Yu and Clack have demonstrated
\footnote{PolyGP: A Polymorphic Genetic Programming System in Haskell}
the utility of type systems in producing more viable offspring.
By limiting mutations to those that result in valid programs, evolution within a
typed language can be accelerated. However, PolyGP only uses three types and a
specialized sublanguage to perform its genetic programming simulation.

I propose a genetic programming framework written in the Haskell programming
language that will evolve genetic programs also written in haskell.
The simulation will make type-safe mutations with whatever functions are
available from a prelude of pre-defined operations and fragments of other
individuals in the population.
For some experiments, the prelude could even include the source of the
simulation itself with the hope that such tools will accelerate convergent
evolution.

From this research, I expect to produce:
\begin{itemize}
\item
    a genetic programming framework for selection, variation, and inheritance
\item
    an analysis of how initial conditions and resource limits affect program
    evolution
\item 
    generated solutions for different types of fitness landscapes
\end{itemize}

\end{document}
